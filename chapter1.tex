\chapter{Introduction}
\label{chap:introduction}

\todo[inline]{program correctness by construction (from specifications to programs); type theory (unification of logic and computation and/vs types as classification/specification); new direction of program derivation, while inheriting problems; a study not really emphasising practicality}

Martin-Löf's intuitionistic type theory \citep{ML-TT73, ML-TT84, Nordstrom-programming} and dependently typed programming~\citep{Altenkirch-why-dependent-types-matter, McBride-Epigram} using the \Agda\ language \citep{Norell-thesis, Norell-Agda, Bove-dependent-types-at-work}.
Intuitionistic type theory was developed by Martin-Löf to serve as a foundation of intuitionistic mathematics, like Bishop's renowned work on constructive analysis \citep{Bishop-analysis}.
While originated from intuitionistic type theory, dependently typed programming is more concerned with mechanisation and practicalities, and is influenced by the program-correctness-by-construction movement.
It has thus departed from the mathematical traditions considerably, and deviations can be found from syntactic presentations to the underlying philosophy.
